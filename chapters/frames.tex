\section{Разработка и реализация фреймов}

Структуры фреймов в OpenJDK являются ключевым элементом ABI всего интерпретатора и определяют такие вещи, как выбор и способ сохранения необходимой информации во время интерпретации методов; способ взаимодействия Java методов, как друг с другом, так и с JNI и Runtime методами, вызываемыми в процессе интерпретации; семантику взаимодействия со стеком вычислений, локальными значениями и аргументами. Правильное создание данных структур должно опираться как на нативную реализацию структуры фреймов, чтобы сохранить совместимость с нативными вызовами и инструментами отладки, так и на создание и использование этих структур в интерпретации, чтобы обеспечить максимально быструю работу приложения.


\subsection{Разработка структуры фреймов}

Первой задачей разработки структуры фреймов было нахождение или составление описания нативных фреймов для языка C. Данные, найденные в онлайн лекции Parallel \& Distributed Operating Systems Group \cite{lecture:frames}, не являются официальным документом от разработчиков компилятора языка C, поэтому их необходимо было проверить, для чего использовался метод обратной инженерии. Были написаны различные программы на языке C, в которых были отражены ключевые сценарии вызовов функций, передачи аргументов и работы с локальными переменными. Далее, данный код был скомпилирован с помощью RISC-V GNU GCC \cite{riscv:gnu} компилятора в ассемблерный код RISC-V, после чего в нём были выявлены места, отвечающие за создание и заполнение фреймов. Семантика данного кода полностью соответствовала найденной структуре, и было принято решение использовать её в качестве основы для создания структуры Java фреймов. 

Проектирование структуры Java фреймов для RISC-V было основано на структуре фреймов для архитектуры PowerPC, однако существенные различия в нативных фреймах данных архитектур оказали соответственное влияние, а именно: нативный ABI RISC-V фреймов содержит только адрес в памяти, указывающий на начало \textbf{предыдущего} фрейма(началом фрейма будем считать последний байт перед этим фреймом), и адрес возврата для \textbf{текущей} функции, в то время как у PowerPC хранится начало \textbf{текущего} фрейма, адрес возврата \textbf{вызываемой} функции, имеется зарезервированное место под первые 8 аргументов и все эти значения лежат внизу фрейма, а не наверху, как в RISC-V.
Данные отличия, а также наличие регистра fp (от Frame Pointer), указывающего начало текущего фрейма, повлекли изменения в реализации методов, которые помогают осуществить просмотр данных фрейма и навигацию между фреймами во время исполнения внутренних Runtime вызовов.

    Одной из особенностей архитектуры RISC-V является обязательное выравнивание вниз на 16 байт значения в регистре sp (от Stack Pointer), которое должно указывать на последний байт текущего фрейма. Во время исполнения кода необходимо соблюдать данное выравнивание, что означает выбор одного из трех решений:
\begin{enumerate}
    \item Применять выравнивание ко всем данным, выкладываемым на стек.
    \item Помнить невыровненное смещение относительно sp, при работе со стеком сдвигать и его, и значение в sp.
    \item Создавать фрейм фиксированного размера и выкладывать все данные внутри него.
\end{enumerate}

Первый вариант нам не подходит из-за того, что размер данных на стеке увеличится примерно в 2 раза, что является неоправданной тратой памяти. Решено было выбрать третий, так как хоть он и требует небольших накладных расходов памяти на поддержание места под значения, которые не всегда необходимо сохранять в памяти, однако, в отличие от второго, он не усложняет работу со значениями на стеке, что может существенно замедлить процесс интерпретации.
% TODO написать про то, что остальные поля заполнялись и менялись исходя из нуждн реализации
% TODO вставить ссылку на приложения со структурой фрейма

\subsection{Реализация входного фрейма}

Самой первой точкой входа в интерпретатор является генерация входного фрейма, который, как и все остальные фреймы, написан с помощью нашего ассемблерного генератора. 

Для создания входного фрейма необходимо рассчитать и выделить под него место на стеке и заполнить все структуры, отображенные на \figref{входной_фрейм}, актуальными данными. Для заполнения нативного ABI необходимо сохранить в верхние два слова значения регистров ra (от Return Adress) и fp, которые указывают на адрес возврата функции и на начало предыдущего фрейма соответственно. "Сохраненные регистры" содержат в себе значения всех неизменяемых регистров, кроме fp и sp, это необходимо для соблюдения конвенции о вызовах C \cite{riscv:convention}, а именно, соблюдение правила о том, что неизменяемые регистры должны иметь одно и то же значение до и после вызова функции. Значение sp регистра будет сохранено в регистре fp, а значение fp мы сохранили в нативном ABI, поэтому их значения не нужно сохранять дополнительно. Локальные данные содержат несколько значений, доступ к которым может потребоваться во время интерпретации, а исходящие аргументы - это аргументы, пришедшие из C вызова, которые необходимо перекопировать на стек для совместимости разных вызовов.
 
\myfig{entry_frame.png}{Структура входного фрейма}{входной_фрейм}{0.5}
 
Во время заполнения данных структур возникали трудности с расставлением правильных смещений относительно fp регистра для точного определения начала структур. Также было необходимо правильно выбрать регистры для всех промежуточных расчетов, чтобы не получилось перекрытия данных. Помимо этого, на данном этапе появилась необходимость загрузки констант в регистры, так как архитектура RISC-V не предусматривает загрузку 64-битной константы в регистр одной инструкцией, и для этих целей нужно написать свой генератор, который по константе сгенерирует оптимальный набор инструкций, загружающих в регистр эту константу. Данный генератор уже написан в GCC, однако найти этот код в огромной системе не получилось, поэтому было решено написать собственный.

\TODO{Нужно ли писать более точные технические подробности про генератор?}

\TODO{Нужно ли указать, что генератор писался коллективно?}



\subsection{Реализация Java фрейма}
При вызове Java метода происходит создание Java фрейма. На предыдущем фрейме должны быть выложены аргументы, а регистр s7, который мы выбрали и назвали esp (от Expression Stack Pointer), и который указывает на первый свободный слот стека вычислений\footnote{Авторский перевод термина operand stack из JVMS}, указывает на место, куда необходимо выложить локалы. В зависимости от размера стека вычислений предыдущего фрейма, занятого на нём места и количества аргументов и локалов текущего метода, предыдущий фрейм необходимо увеличить, чтобы вместились все данные, либо уменьшить, чтобы не занимать лишнего места на стеке, которое мы не будем использовать во время интерпретации текущего метода, для этого нужно соответствующим образом сдвинуть регистр, указывающий на начало текущего фрейма, запомнив конец предыдущего фрейма до смещения в специально отведенном месте не стеке, чтобы восстановить его первозданный вид после выхода из метода.\newline

\myfig{Ijava_frame.png}{Структура Java фрейма}{java_фрейм}{0.45}

К сожалению, структуру фрейма не удалось сильно видоизменить для экономии места или регистров. Были идеи развернуть аргументы и локалы, что позволило бы упростить доступ к ним, но из-за устройства байткодов пришлось бы каждый раз перекопировать аргументы, что сильно замедлит вызов. Место под мониторы нельзя выделить заранее, а также нельзя его передвинуть в другое место для ускорения расширения стека при добавлении монитора, отведенное ему место является оптимальным по всем параметрам.




\subsection{Реализация нативного Java фрейма}
В нативном вызове нет operand stack, а место под аргументы рассчитывается иначе, исходя из нативного соглашения о вызовах. Это соглашение описано менее четко, чем для Java, и для того, чтобы точно его реализовать необходимо было вновь прибегать к методам обратной инженерии. По этому соглашению также происходит заполнение регистров и стека аргументами. Для полной уверенности было проведено тщательное тестирование, о котором мы поговорим позже.

\myfig{native_frame.png}{Структура нативного Java фрейма}{нативный_java_фрейм}{0.6}

При нативных вызовах делается также намного больше проверок и обработок исключений. Помимо этого необходимо корректно обработать возвращаемое значение, пришедшее из Си функции. Например, по спецификации Java у значения bool должно быть значение 0 или 1, поэтому необходимо делать преобразования ненулевых значений.


