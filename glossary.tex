\newglossaryentry{аргумент} {
    name=аргумент,
    description={Значение, передаваемое функции, а также символьное имя (название переменной) в тексте программы, выступающее в качестве идентификатора этого значения}
}

\newglossaryentry{байткод} {
    name=байткод,
    description={Инструкция языка программирования Bytecode}
}

\newglossaryentry{локал} {
    name=локал,
    description={Локальная переменная функции, то есть переменная, объявленная внутри функции}
}

\newglossaryentry{функция} {
    name=функция,
    description={Это поименованная часть программы, которая может вызываться из других частей программы столько раз, сколько необходимо}
}

\newglossaryentry{метод} {
    name=метод,
    description={Функция, принадлежащая какому-то классу или объекту}
}

\newglossaryentry{порт} {
    name=порт,
    description={Версия программы, адаптированная под целевую платформу (операционную систему, процессор), либо модуль этой программы, служащий для данной адаптации}
}

\newglossaryentry{регистр} {
    name=регистр,
    description={Ячейка сверхбыстрой памяти процессора}
}

\newglossaryentry{фрейм} {
    name=фрейм,
    description={Область памяти или структура данных, содержащая некоторую информацию, необходимую для исполнения конкретной функции}
}

\newglossaryentry{интерпретация} {
    name=интерпретация,
    description={Процесс исполнения кода одного языка при помощи кода, написанного на другом языке}
}



% \begin{itemize}
%     \item аргумент функции --- значение (число, указатель и т. д.), передаваемое функции, а также символьное имя (название переменной) в тексте программы, выступающее в качестве идентификатора этого значения. 
%     \item байткод --- инструкция языка программирования Bytecode
%     \item интерпретация
%     \item локал
%     \item метод
%     \item порт
%     \item регистр
%     \item стек 
%     \item стек вычислений
%     \item тип данных
%     \item фрейм
% \end{itemize}